\section{Colorizar}

\subsection{Código C}
	El código C se trata de una conjuncion de Fors, el exterior que recorre desde la segunda fila hasta la ante ultima,  y el interior que recorre desde la segunda columna hasta la ultima, dejando asi afuera a todos los bordes, tal como el enunciado decia. Luego en cada iteracion del ciclo interior, que es donde se hace las opearciones que modifican la imagen, lo que hacemos es crear un arreglo de unsinged chars, "res", que es  donde guardamos los maximos de cada canal en comparacion a todos  los pixeles lindantes del pixel en el cual estemos parado.
\begin{itemize}
\item {Res $[$0$]$ $\leftarrow$ MaximoLindantesAzul}
\item {Res $[$1$]$ $\leftarrow$ MaximoLindantesVerde}
\item {Res $[$2$]$ $\leftarrow$ MaximoLindantesRojo}
\end{itemize}
Luego con estos tres valores calculamos el alpha correspondiente de cada canal por el cual voy a multiplicar a cada uno. Y por ultimo reescribimos el pixel final, en la imagen source con cada canal multiplicado  por dicho alpha.

\subsection{Código ASM}
	El código en ASM se trata tambien de una conjuncion de ciclos. El recorre las filas desde la segunda hasta la anteUltima, y el interior recorre las columnas desde la segunda hasta la anteultima, pero saltando de a dos pixeles, que es la cantidad que procesamos simultaneamente con instruccciones SSE. 
    El ciclo interior consta de tres partes, la primera es tan pronto se levanta de memoria los pixeles a procesar y todos sus lindantes, calculamos en dos registros los maximos de cada canal, de ambos pixeles a procesar con respecto a todos sus lindantes, y los guardamos ambos en un registro xmm. Luego la segunda parte es calcular el maximo de los maximos de ambos al mismo tiempo. Luego atravez de una par de operaciones, muy especificas para explicar, logro tener un registro xmm de dw, con cada dw representando un canal de cada pixel, en el orden establecido (argb), donde tengo un 1-alpha en la posicion del canal que no teine el maximo de los maximos, y un 1+alpha en la posicion que tiene al maximo de los maximos, y luego concluyo multiplicando a cada pixel por su registro con los alphas indicados, lo reduzco a tamaño de 32 bits por pixel, los uno y los escribo en el destino.
	
	
\subsection{Experimentación}
\subsubsection{Idea}	En la experimentacion de este filtro al igual que en el resto vamos a comparar el rendimiento respecto a los ciclos de clock, que tiene la funcion colorizar en C desde -o0 a -o3 contra asm. \\ Sin embargo luego vamos a contrastar la funcion de asm, contra sigo mismo pero primero agragando jumps de forma que no influya el flujo del programa solo para molestar al jump predictor. Luego vamos a correrlo tal cual esta, y de apoco vamos a ir desenrollando el codigo. Como dijimos tiene un ciclo externo y uno interno, por lo que vamos a desenrrollar primero el interno 2 veces, luego 4,16,32 y ver que pasa, y finalmente vamos a saltar a desenrrollarlo completamente, cosa de no dejar casi ninguno controlador de flujo.
	   
\subsubsection{Hipótesis}
	Nuestra hipotesis es que el rendimiento va a ir mejorando a medida que vayamos cambiando los programas respectivamente a como los fui mencionando, osea el mas lento va a ser el codigo de asm, molestando al jmp predictor, y el mas rapido el desenrollando el codigo 32 veces. El ultimo test, por mas que desenrollemos todo el programa creemos justamente que va a ser el mas lento, por el tamaño del codigo, creemos que al hacer un codigo de tamaño mayor al de la cache, en un momento el pc va a estar haciendo muchisimos mas miss en la cache, que los que ahorra sacando los controladores de flujo.
	
	
\subsubsection{Resultados}
	Graficos lindos de lucia :D
	ciclos promedio colorizar c -o0 100 iteraciones : 101.898.160
	ciclos promedio colorizar c -o3 100 iteraciones : 21.567.776
	ciclos promedio colorizar original 1000 iteraciones: 5.980.263
	ciclos promedio colorizar JodiendoJumps1 1000 iteraciones: 5.977.453
	ciclos promedio colorizar Unrolling Ocho ciclos 100 iteraciones : 5.873.712
	
	
	
	ciclos promedio colorizar Unrolling lena 1024x768
	 todos los cilcos 100 iteraciones : 35.285.958
	
	ciclos promedio colorizar original lena 1024x768 100 iteraciones:	21.050.994
	
	
	
\subsubsection{Conclusión}
Conclusion Todo es una mierda.
Conclusion posta:
 Primera gran conclusión reveladora, c -o0 demuestra ser mucho mas lento que optimizando, el código mejoro 5 veces su cantidad de ciclos por llamada al compilar con -O3.
  Luego sin embargo la función hecha en  asm le sigue ganando por bastante a C, en una proporción de 3.5 veces mas rápido en cantidad de ciclos todavía.
  Ahora encuanto al experimento personalizado, hicimos una comparación con distintos códigos para medir la influencia del jump predictor en la ejecución del código. Primera prueba fue ejecutar el mismo código que teníamos con el agregado de que en el ciclo interior (el que se repite mas veces) metíamos una serie de saltos transparentes con diferentes comparaciones, para tratar de romper con pipeline del procesador. Sin embargo por mas duro que itnentasemos descubrimos que el algoritmo que dirige al jump predictor es bastante mas sofisticado de lo que pensábamos, por lo que no pudimos influenciar en lo mas mínimo la eficiencia del código. 
   Luego para contrastar el ultimo experimento mencionado, intentamos implementar lo contrario, desenrollar el código de manera que nunca influya un miss hit del jump predictor, en la ejecución del programa, sin embargo ya sacando la conclucion de lo mencionado recién, de que el sistema que lo regula es mas sofisticado del esperado, el desenrollar el código principal 2,4,8 veces, no estuvo mostrando ninguna diferencia con respecto al original. 
   Por ultimo pasamos a otro experimento con desenrollar, en el cual se pretendia desenrollar completamente el codigo, generando un resultado en caunto a la cantidad de ciclos consumidos por llamada mucho mayor, por el tema de que un codigo tan grande no entraria completametne en la cache porl o que a partir de un punto el PC estaria generando constatnes miss hits en la cache y contrarestrando lo logrado evitar por el jump predictos. Y De hecho este experimento funciono muy bien, lo intentamos con una imagen mayor de lena, de 1024x768 para que el ciclo pueda ser desenrrollado mas cantidad de veces, y asi se ve que el codigo modificado tardo en promedio al rededor de unos 14.000.00 de ciclos mas que el original.
